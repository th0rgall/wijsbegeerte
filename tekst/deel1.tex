\section*{\centering \underline{Deel 1: Interesses, instrumenteel en intrinsiek}}

\section{De ervaringsmachine}
\subsection{De droommachine}
\newtheorem*{-R}{Principe -R - Alles is ervaring}
	\begin{-R}
	\textbf{Enkel} de ervaring die een object kan bieden bestaat of heeft belang.
	\newline Deze is onbetwistbaar echt, immanent.
	\end{-R}
	\begin{itemize}
	\item Solipsisme: de filosofie dat er maar een enkel bewustzijn bestaat: dat van de waarnemer. (egocentrisch)
	\end{itemize}
\newtheorem*{+R}{Principe +R - Ervaring is niet alles}
	\begin{+R}
	Wij verlangen te ervaren van het fetisjistisch object.
	\end{+R}
	\begin{itemize}
	\item Fetisjisme: Het verschil tussen het echte en het perfecte duplicaat is niet ervaarbaar. De authenticiteit van een object is dus niet te beschrijven, maar het is transcendent, magisch.
	\end{itemize}


\begin{tabular}{p{7cm} p{6cm}}
	\hline
	
	 \raggedright Mephisto (-R) & Maniphesta (+R)\\
	\hline
	\begin{flushleft}
	\begin{enumerate}
	
	\item[Arg. 1:] nieuwsgierigheid
	\item[Arg. 2:] geen negatieve bijwerkingen, \\geen kater
	\item[Arg. 3:] unieke ervaring
	\item[Arg. 4:] ultieme genieting
	\item[Arg. 5:] antwoord: onechte ervaring \\bestaat niet
	\end{enumerate}	 
	\end{flushleft}
	
	&
	\begin{flushleft}
	
	\begin{enumerate}
	\item[Tegenarg. 1:] vereiste van onvoorspelbaarheid
	\item[Tegenarg. 2:] onbeslist, prijs = solipsisme; egocentrisch
	\item[Tegenarg. 3:] o.k. maar te makkelijk, geen verdienste
	\item[Tegenarg. 4:] neutraal, voorspelbaar, verveling
	\item[Tegenarg. 5:] aanval: zelfbedrog!
	\item[Tegenarg. 6:] weerlegging: echte ervaring $\neq$ \textbf{ervaring van het echte!}
	\end{enumerate}
	\end{flushleft}


	\\
	
	\hline
	
\end{tabular}
\\
Paradox: enerzijds verlangen we enkel naar ervaring (-R), anderzijds ook naar het fetisjistisch object (+R).

\subsection{Alles is ervaringsmachine}
Mephisto's meesterzet: we zitten al in een ervaringsmachine die de grilligheden van het leven simuleert. Ook steken we ons leven vol met ervaringsmachientjes. \textsc{Rousseau}: we kunnen niet leven zonder techniek. Het echte, de pure natuur, is niet te scheiden van de cultuur. Dit definieert als het ware de mens, we willen, we zijn gedoemd tot beheersing.

\paragraph{Lectuurvb. II.a: het echte water} \textsc{De Saint-Exup\'ery}, De kleine prins
\\ \\
De kleine prins wijst de dorstlessende pil af. Er wordt gesuggereerd dat de techniek eigenlijk maar illusionaire vooruitgang is, er worden nieuwe behoeften gecre\"eerd. Dan krijgt hij voldoening door het "echte ervaring" van het water te drinken, afkomstig van de gegevens die hij als echt beschouwd: naar de waterput lopen, de katrol, de moeite.

\paragraph{Lectuurvb. II.b: de echte croisson} \textsc{De Saint-Exup\'ery}, Terre des hommes
\\ \\
Het normale ontbijt na een bijna-doodservaring (bijna-crash) is een echte ervaring. Dit kwam door de onzekerheid dat ze het gingen halen, de onbeheersbaarheid, de verrassing, het geloof in het leven.

Herformulering van de paradox hierboven:

\begin{itemize}
	\item We verlang het beheersbare: hetgeen aan ons verlangen beantwoordt. 
	\item We verlangen het onbeheersbare: hetgeen niet aan ons verlangen beantwoord, is fascinerend.
	\\ Persoonlijk voorbeeld: als je een serie gaat kijken, zet je een ervaringsmachientje op. Misschien wil je dat de structuur net zoals altijd is (F.C. De Kampioenen), of misschien wil je verrast worden (Game of Thrones).
\end{itemize}

\section{De eerste hypothese}
\begin{center}
\begin{huge}
Alles is beheersing
\end{huge}
\end{center}
h1 - h2: alle interesse is uiteindelijk beheersingsinteresse.

\subsection{Alles is techniek}
Praktische beheersing van de wereld neemt twee vormen aan:
\begin{itemize}
\item Technische maakbaarheid
\item Overlevingswaarde
\end{itemize}
Technische interesse: het maken en hermaken van de wereld is wat ons bezighoudt, ook indien niet alles effectief maakbaar zou blijken.
\subsubsection*{Machineconcepten}
Machinerevolutie doorheen de geschiedenis:
\begin{enumerate}
\item Eerst werd menselijke arbeid werd omgezet in andere energie, daarna konden alle vormen van energie naar elkaar omgezet worden.
\item Een opvatting van de mens als hypercomplexe neuronale en genetische machine ontstond..
\item Ontstaan cognitieve kennismachine: computer.
\end{enumerate}
Kan een machine denken of is wat wij doen als we denken een machinaal gebeuren. De kwestie van \textit{mind en brain} gaat voortaan over de driehoek tussen het biologische, het psychische en het machinale.
\subsubsection*{Het paradijs van de techniek}
Technische optimisten dromen van een technische utopie die onze onbeperkte behoeften vervult. Voor elk oplosbaar probleem is de oplossing een technische oplossing. De maatschappij is gericht op het continu verder beheersen en controleren. Ook onze natuur zelf is maakbaar, we willen beschikken over geboorte en sterven.
\subsubsection*{De paradox van de techniek}
We \underline{worden beheerst} door
\begin{itemize}
\item het instrument dat we uitvonden om de natuur te beheersen
\item onze beheersingsdrift zelf, onbeheersbaar geworden
\end{itemize}
Een schakel die uitvalt in ons technisch netwerk zal ons uitschakelen. 
\\Martin \textsc{Heidegger} (1889-1976): "voor elke technische oplossing duikt een nieuw (en groter) probleem op."
\subsection{Alles is selectie / overleving}
De meest fundamentele praktische interesse is overleven, we zijn overlevingsmachines. De tweede versie van de eerste hypothese: alles wat ons zou kunnen interesseren buiten de overleving is ofwel een onrechtstreekse bijdrage tot de overleving, ofwel een uitbreiding ervan: het verbeteren van onze levenscondities zoals in de ``alles is techniek''-versie.
\\
$\rightarrow$ ingebakken in de common sense over leven
\\
\\
\textsc{\textbf{Aristoteles}} (\& ME): eeuwig bestaan, aan zichzelf gelijkblijvende soorten.
\\ Vanuit pragmatische observatie: o.a. ook het cyclische van de seizoenen.
\\
\textbf{\textsc{Newton}}: ontstaans- \& ontwikkelingswetten bestaan niet, ernaar zoeken is metafysisch en theologisch.
\\ Pas 18\textsuperscript{e} \& 19\textsuperscript{e} eeuwse tijdgenoten en voorlopers van \textsc{Darwin} hadden evolutiegedachten: \textsc{Buffon}, \textsc{Maupertuis}, \textsc{Goethe}, \textsc{Lyell}, \textsc{Lamarck} en \textsc{Paley}.
\\ \textbf{Darwin} is echter de eerste die een reden kan geven, de evolutie van de levende soorten is geleid door \emph{natuurlijke selectie}:
\\
\begin{center}
\begin{tikzpicture}[shorten >=1pt,node distance=2cm,on grid,auto] 
\node (a) at (0,-9.5) {\textbf{Mutaties}};
\node (a2) at (0,-10) {\footnotesize{toeval \& statistiek, later genetisch verklaard}};
\node (b) at (0,-11) {\textbf{Variaties}};
\node (b2) at (0,-11.5) {\footnotesize{$\Rightarrow$ verschil in ``designkwaliteit'', zonder designer}};
\node (c) at (5,-12.5) {\textbf{\large{Evolutie}}};
\node (d) at (0,-13) {\textbf{Selectie}};
\node (e) at (0,-14) {\textbf{Adaptatiegraad} = fitness, reproductief succes in omgeving};


\path
(a2) edge[->] (b)
(e) edge[->] (d)
%(b2) edge[->] (c)
(b2) edge[->, bend right, out=-60, in=-120] (e)
(d) edge[->] (c);
\end{tikzpicture}
\end{center}
$\Leftrightarrow$ voorganger/tijdsgenoot \textsc{Lamarck}: soorten evolueren door hun \emph{verworven} eigenschappen \\ \indent(vb: Lamarck: de lange nek van giraffen $\Leftrightarrow$ Darwin: kinderen verschillen in lengte)
\\
$+$ naast adaptaties, zeggen \textsc{Gould} en \textsc{Lewontin} dat veel evolutie door \textbf{exaptatie} voortkwam (= onvoorziene maar gunstige neveneffecten van veranderingen), vb. vleugels van vogels (origineel warmteregulerende functie). Deze imperfecties spreken \emph{intelligent design} tegen.
\\ \textbf{Sociobiologie:} (nu \textit{evolutionaire psychologie}) is een versie van het neodarwinisme waarbij men een volledige verklaring van de cultuur en de menselijke ervaring in het verlengde van de darwinistische categorie\"en van overleving, selectie, mutatie, variatie en reproductief succes op het oog heeft.

\subsubsection*{Neodarwinisme, the Unit of Selection}
Wat is dan de verklaring voor empathie \& co\"operatief gedrag? 

\begin{enumerate}
\item \textbf{Synthetische evolutietheorie} \textsc{Mendel} (1866) \& \textsc{De Vries} ontdekken basis van erfelijkheid. Gelinkt met Darwin geeft dit in in 1930 een aanzet tot het ``neodarwinisme''
\item \textbf{Nieuwe synthese} \textsc{Crick \& Watson} (1955): ontdekking DNA.
\item 1960: twijfels over de \emph{Unit of Selection}, altru\"isme verklaren.
	\begin{enumerate}
	\item \textsc{Darwin} \& co: het \emph{individu} is die unit.
	\item \textsc{Lorenz}: vb. moederinstinct, de unit is de \emph{species}, maar er wordt geen verklaring gegeven
	\item de \emph{kin} groep, (persoonlijk bv. stammen in oorlog)
	\item vanaf 1970: \textsc{Trivers}, \textsc{Hamilton}, \textsc{Smith}: \textbf{Gene Selection}: genen willen overleven, wij zijn slechts genenmachines, \underline{beheerst} door genen. Verband met sociobiologie \& evolutionaire psychologie.
	
	\end{enumerate}
\end{enumerate}

\begin{itemize}

\item Met behulp van bv. de game theory van \textsc{Nash}, simulaties van belonings/bestraffingsspellen in 1984 (voorbeeld van het \emph{Prisoner's Dilemma}) kan aangetoond worden dat genen uiteindelijk moeten kiezen voor co\"operatie om daar uiteindelijk evolutionair voordeel uit te halen.

\item \textsc{Dawkins} synthetiseerde en populariseerde de gene selection, en breidde het uit met \textbf{memen}: de eenheden van reproductie in cultuur, zoals het gen dat is voor de natuur. Gedrag, muziek, ... dat door middel van \emph{imitatie} overgeleverd wordt. Hier ook gelden Darwins principes van evolutie en overleven (mutatie, variatie, selectie, evolutie). Dit was even de standaardgedachte, maar nu amper door kritiek:
	\begin{itemize}
	\item \emph{embodiment} kwestie: het gen zit in het DNA, maar waar zit de meme in? neuronen?
	\item is dit niet ``lamarckiaans''? Nieuwe cultuur kan verworven en doorgegeven worden.
	\item Genen: mutatie is de uitzondering (traag) Memen: mutatie is de regel (snel).
	\item bv. Overtuigingen: hun selectie is doelgericht, en een nieuwe overtuiging is juist niet geadapteerd aan de omgeving.
	\end{itemize}
\item Hierna loopt alles door elkaar. Er zijn heel veel neo- / evolutionaire $\Psi$ theorie\"en. In 2.3 vooral evolutionisten die vragen stellen bij de principes van Darwin.
\end{itemize}


\subsection{Morele intu\"ities}
O.a. \textsc{Wilson} en andere sociobiologen stellen dat alle hooggestemde culturele, morele waarden directe of indirecte (in vermomming) oververlevingswaarden zijn. Kritiek:
\begin{itemize}
\item de ``ontdekking'' van evolutionair voordeel in co\"operatie is een \emph{voorwaarde} voor altru\"isme, geen gesloten reden tot altru\"isme.
\item Er zijn extreme en minder extreme voorbeelden van zelfopoffering. Is er een kern in altru\"isme die pas onstaat wanneer we echt ons eigenbelang wegschuiven? Dit om te voldoen aan een intrinsieke eis (los van Darwinistische principes) van bv ``menselijk zijn'' (lectuurfragment Primo Levi, is dit een mens?). 
\item De verklaringen van sociobiologen worden opgemaakt in casu en kunnen zich niet toetsen aan alle verschijnselen.
\end{itemize}

\subsection{Taal: Venus van de sterren, emergente symbolen}
 \begin{itemize}
 \item Criterium voor taal vs communicatie: we behandelen enkel het \emph{semantische} aspect.
 \item Taal is: 
 \begin{itemize}
 \item een voortzetting van mutaties en selecties (continu - het signaal - dierentaal - sociobiologen)
 \item $\leftrightarrow$ een radicale nieuwe sprong van natuur naar cultuur (emergentie - discontinu - symbolen = nieuw - critici: \textsc{Langer \& Chomsky} o.a.)
 \item \textsc{Langer}: continu opgestegen uit de signaalsystemen van de hogere diersoorten \& discontinu door de emergentie van nieuwe eigenschappen.
 \end{itemize}
 \end{itemize}
 
Helen Keller \& lerares Anne Sullivan, een case study door Susanne \textsc{Langer}:
Helen, doofstom en blind, gebruikt taal eerst als een instrument voor directe beheersing, en ziet daarna de kracht van het symbool in (lectuurfragment van de waterpomp).

\begin{center}
\begin{tikzpicture}[shorten >=1pt,node distance=2cm,on grid,auto] 
\node (a) at (3,0) {\textbf{Teken}};
\node (b) at (0,-1) {Signaal (s)};bestudeerd
\node (c) at (6,-1) {Symbool (S)};
\node (d) at (0,-3) {Gebruiker (G)};
\node (e) at (6,-3) {Symboolgebruiker(G)};
\node (f) at (-2,-4) {Signaalteken (s)};
\node (g) at (1.5,-4) {object (o)};
\node (h) at (4,-4) {Taalsymbool (S)};
\node (i) at (6,-4) {};
\node (j) at (8,-4) {gedenoteerd object (d)};
\node (k) at (6,-5) {connotatie (c)};

\path
(a) edge[->] (b)
(a) edge[->] (c)
(d) edge[-] (f)
(d) edge[-] (g)
(f) edge[->] (g)
(e) edge[-] (h)
(e) edge[-] (j)
(h) edge[-] (k)
(h) edge[->,densely dashed] (j)
(j) edge[-] (k);
\end{tikzpicture}
\end{center}
\begin{description}
\item \textbf{Connotatie:} intensie, \emph{sense}, begripsinhoud, mentale conceptie, "wat"
	\begin{itemize}
	\item \textsc{Frege}: connotaties zijn objectief, vb. $\neq$ definities van een cirkel \\ een \emph{derde aspect}, de \emph{F\"arbung}, brengt de subjectieve associaties in rekening.
	\item \textsc{Langer}: connotaties \& associaties zijn niet te scheiden, er zijn graden van objectiviteit ($\neq$ idee\"en over $H_{2}O$)
	\end{itemize}
\item \textbf{Denotatie:} puur objectief, het referent, omvang van het gezegde of extensie, het "ding"
\end{description}

\begin{center}
\begin{tikzpicture}[shorten >=1pt,node distance=2cm,on grid,auto] 
\node (a) at (0,0) {S   =};
\node (b) at (2,0) {woord};
\node (c) at (0,-1) {c =  };
\node (d) at (2,-1) {begrip};
\node (e) at (0,-2) {d =};
\node (f) at (2,-2) {referent};


\path

(a) edge[->] (c)
(c) edge[->] (e)
(b) edge[->] (d)
(d) edge[->] (f);
\end{tikzpicture}
\end{center}

\subsubsection*{Wat maakt een taalsymbool nu tot een symbool? } (\textsc{Cassirer})
\begin{center}
\begin{tabular}{c c}
\textbf{Signaal} & \textbf{Symbool}\\
stimulus/respons & interpretatie (connotatie)\\
aanwezig object & afwezig, mentaal object\\
natuurlijk, causaal & conventioneel\\
communicatief/expressief & articulerend/expressief\\
\end{tabular}
\end{center}

Articuleren: gedachten en emoties ontwikkelen en transformeren tijdens het uitdrukken


\subsubsection*{Nieuwe theorie}
Het geval van eigennamen:
\begin{itemize}
\item pure denotatie, s $\rightarrow$ d
\item \textsc{Kripke \& Putnam}: S $\rightarrow$ d $\rightarrow$ c
\end{itemize}

\textbf{Besluit:} \underline{connotaties en denotaties zijn noodzakelijk in taal}, ze laten toe een samenhangend verhaal op te hangen over al dan niet aanwezig object.

\subsection{Intrinsieke betekenissen}

\begin{itemize}
\item \textsc{Langer}: apen zitten in het begin van symbolisering, ze kunnen bv. gefascineerd of bevreesd worden zonder praktisch nut
\item Kanttekening bij evolutionaire verklaring: voor welk verschil zorgt onze symboleninteresse? bewustzijn $\rightarrow$ denken; voelen $\rightarrow$ emotie. \\ We kunnen denken, dus we kunnen al het ``goede en slechte'' dat eigen is aan de mens niet wijten aan de natuur of selectie, evolutionaire verklaringen zouden te goedkoop zijn.
\item \textsc{Leibniz}: ``80\% van de mensen zijn dier, de andere 20\% voor 80\% van hun tijd'' we hebben zowel een symbolische interesse als dierlijke driften en behoeften. Deze leven samen en botsen soms.

\end{itemize}
De eerste hypothese (h2, de maakbaarheid en de overleving) stelt de interesses een \underline{instrumenteel karakter} hebben: er moet een praktisch nut voor de mens zijn. Niets is een waarde op zich.
\\
$\Leftrightarrow$ de intu\"itie dat er \underline{intrinsieke waarden} bestaan: \\ - ofwel zijn symbolen zelf intrinsieke betekenissen zoals bij bv. een kunstwerk \\ - ofwel zijn symbolen de perfecte en laatste instrumenten om tot het intrinsieke doel te komen, bv. kennis
\section{De tweede hypothese}
\begin{center}
\begin{huge}
Alles is kennis
\end{huge}
\end{center}
De drang die ons voortdrijft is niet die om alles te kunnen (de wereld te beheersen), maar die om alles te kennen (cognitieve beheersing van de wereld), te begrijpen en te doorgronden. De structuur:
\begin{itemize}
\item het schema van \textbf{Comte}, een algemene behandeling over de evolutie van kennis (h3)
\item behandeling vanuit de \emph{geschiedenis}:
	\begin{enumerate}
	\item \textbf{De Oude Alliantie} (Klassieke Oudheid \& ME) Kennis \& Zingeving 
	\item \textbf{De Nieuwe Alliantie} (Moderne tijd) Kennis \& Beheersing
	\end{enumerate}
	
\end{itemize}
\subsection{Het schema van Comte}
Auguste \textsc{Comte}
\begin{itemize}
\item $19^e$ eeuw, uitloper van de verlichting
\item vader van het positivisme.
\item Vertolker van een ondertussen al voorbijgestreefd (zie verder) standpunt over de evolutie naar wetenschap en zijn belang in de cultuur. \\ Zijn schema onderscheidt een \emph{initieel stadium, een overgangsstadium en een finaal stadium} zoals in de ontwikkeling van het individu (ontogenese | kinderlijk naar volwassen denken) en die van een soort/cultuur (fylogenese | primitieve naar gesofisticeerde culturen).
\end{itemize} 
\begin{enumerate}
\item \underline{\textbf{Het theologische of mythische stadium:}}
\\ \emph{Verbeelding} staat centraal.
De mens stelt de meest onoplosbare vragen zoals naar de oorsprong van alles en gebruikt in zijn verklaring onzichtbare bovennatuurlijke wezens die een menselijke wil hebben, en wiens daden de verklaring zijn.
\\
Drie deelstadia:
\begin{enumerate}
\item \underline{Het fetisjisme of animisme:} gevoel/instinct\\
Alles in de natuur krijgt een ziel. Die zielskracht is niet gelijk verdeeld, hij concentreert zich bv. in een boom, een bron... met hemellichamen als toppunt (bij de Grieken).
\item \underline{Het polythe\"isme:} toppunt van verbeelding \\
Het leven wordt aan de natuur onttrokken, het sacrale belichaamt zich nu in godheden, wezens met een persoonlijk karakter zoals in de Griekse opvatting die de natuur beheersen (theologie!).
\item \underline{Het monothe\"isme:} logica komt op\\
Tussenstadium: dualisme, maniche\"isme (god van goed \ god van slecht). Men verkiest een absolute goddelijke wezenheid als enkele bron van alles boven een een chaotische veelheid.
Volgende stap: valt de orde en de wetmatigheid van het universum die een godheid lijkt te vereisen niet nog anders te bekijken?
\end{enumerate}
\item \underline{\textbf{Het metafysische of abstracte stadium:}}\\
Overgansfase: bovennatuurlijke wezens worden vervangen door essenties. Maar zijn nog steeds afkooksels van de eerste fase.\\
Zichtbare en veranderlijke fenomenen worden verklaard in termen van onzichtbare en onveranderlijke essenties. 
Kritiek van Comte op deze denkwijze: beantwoordt de concrete realiteit wel aan onze abstract bedachte essenties?
\item \underline{\textbf{Het positieve of wetenschappelijke stadium:}}\\
"De positieve filosofie" + empirisme (zintuigelijke ervaring). Wetten zijn algemene feiten, constante realties tussen de waargenomen feiten. Metafysica heeft geen te kort aan denken maar aan empirische controle. Denken is niet hetzelfde als kennen.  Mythe is een soort primitieve wetenschap, die nog geen cognitief significate aantwoorden kan geven maar wel al de cognitieve vragen stelt (schepping, ontstaan heelal).
\item[] \underline{\textbf{Besluit:}}
\begin{enumerate}
\item Wetenschap is meer en minder: de kinderlijke naïviteit moet in zekere zin verloren gaan om plaats te maken voor echte kennis.
\item Wetenschap is in zekere zin een zoektocht naar verklarende wetten maar dat deed de metafysica ook al. We moeten ons beperken tot controleerbare wetmatigheden. De echt diepe verklaringen zijn te metafysisch.
Volgens Comte zoeken we naar zo groot mogelijke veralgemeningen en unificaties binnen elk domein en tussen de domeinen.
\item De mythe is volgens Comte ook van de orde van de kennis ( zij zoekt immers naar antwoorden op grote vragen) veeleer dan het verhaal over de stichting of de ordening van de gemeenschap. De mythe is een soort primitieve wetenschap.
\end{enumerate}
\end{enumerate}
\subsection{Mythe en wetenschap: voorlopige bedenkingen}
Comte's positivisme = sci\"entisme: een onwrikbaar geloof in de wetenschap en het geloof dat de cultuur slechts door de wetenschap vooruitgaat. Een verlichtings en huidige gedachte: bij de mythe werden de vragen al gesteld, men zag en ziet mythische en religieuze overtuigingen als primitieve pogingen om tot een rationeel systeem van wereldverklaring te komen die bij gebrek aan de juiste middelen niet konden lukken.
\\ \\
Een verdere algemene verlichtingsgedachte is de volgende: de mens komt tot het bedenken van de mythe vanuit de drang om de onzekerheid van zijn lot in een gevaarlijke wereld te bemeesteren. Die onzekerheid resulteert in de gedachte aan goden die het wereldgebeuren regeren. En die onzekerheid komt dan weer uit onwetendheid. De verlichters (\`a la Comte) meenden dat wanneer de mens de waarheid over de natuur zou kennen de angst voor de goden zou weggenomen zijn en ook het geloof erin. Voor de verlichters is de mens een rationeel wezen voor wie alles een kwestie is van waarheid en inzicht. Hef de onwetendheid op en het licht zal schijnen in de duisternis.
Maar is de zin van kunst, religie... : wel gelegen in zo'n soort rationaliteit?
\subsection{Kanttekeningen bij een clich\'e}
Blijkbaar had ook Comte zijn twijfels over de verdwijning van de zin-vraag wanneer we eenmaal tot ware wetenschap zouden zijn gekomen. Daarom wilde hij de grondslagen leggen van een wetenschap van de maatschappij: de sociologie. Deze wetenschap zou ons laten zien welke wetten het samenleven echt beheersen en zou de mens bevrijden uit zijn onwetendheid want beschikken over waarheid is beschikken over de middelen om alles te veranderen en te verbeteren.
Volgens Comte had ook de filosofie nog een taak: wegens de specialisatie van de kennis was een synthese van de wetenschappen nodig: filosofie zou een subjectieve synthese, die de betekenis van de diverse wetenschappelijke inzichten voor de mens duidelijk maakt, kunnen aanbieden.
\\
\\
Het is eigenlijk zo dat de positieve filosoof een socioloog is. De sociologie is niets anders dan het inzicht in de sociale orde en geschiedenis die toelaat het schema van de drie stadia op te stellen en ons verleden en onze toekomst te assumeren. Dat een hele maatschappij mythisch, metafysisch enz is opgebouwd betekend in feite dat de motor van het geheel nergens anders te zoeken is dan in welbepaalde sociale praktijken.
\\
\\
\textsc{Comte} is een sci\"entist, geloof in de wetenschap maar geen objectivist. Hij is tegen de materialisatie en de reductionistische interpretatie van de wetenschappen en hun onderlinge verhoudingen. Nieuwe religie - de relegie van de mensheid? Waarin de mens zou ge\"eerd en vereerd worden.
\subsection{Mythe en wetenschap: aanvullende bedenkingen}
Wetenschappelijke kennes als een instrument om een ander soort van vragen te beantwoorden, namelijk de ordening en de verlichting van het leven $\rightarrow$ het is na\"ief om te denken om wetenschappers en sociale ingenieurs ten dienste te stellen van die vragen.
\\
\\
COMTE: mythe $\longrightarrow$ metafysica $\longrightarrow$ wetenschap.
\\
FRAZER: magie $\longrightarrow$ religie $\longrightarrow$ wetenschap.
\\
\\
\begin{center}
\underline{Van mythos naar logos:}
\end{center}
\begin{enumerate}
\item Vooruitgang, evolutionair model
\item Beiden zien mythe/magie als proto-wetenschap
\begin{flushleft}
Coginitief\\
Eerste poging tot verklaring\\
Primitieve logos
\end{flushleft}
\end{enumerate}

\begin{center}
\begin{tikzpicture}[shorten >=1pt,node distance=2cm,on grid,auto] 
\node (a) at (3,0) {\textbf{Twee modellen van denken over de mythe}};
\node (b) at (0,-2) {Als cognitief project};
\node (c) at (6,-2) {Als zingeving};
\node (d) at (0,-3) {Primitieve logos};
\node (e) at (6,-3) {Verbonden met praktijken en rituelen};
\node (f) at (0,-4) {Comte, Frazer, ...};
\node (g) at (6,-4) {Malinowski, Wittgenstein, ...};
\node (q) at (3,-6) {Poging tot verzoening: 'mixed motives' (Horton)};
\node (z) at (3,-6.5) {= mythe gaat vooraf aan de splitsing tussen kennisinteressse en andere interesses.};

\path

(f) edge[->] (q)
(a) edge[->] (b)
(a) edge[->] (c)
(g) edge[->] (q);
\end{tikzpicture}
\end{center}
