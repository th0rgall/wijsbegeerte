\section*{\centering \underline{Deel 2: De oude alliantie}}
\section{Alles is water}

\subsection{Inleiding}
\begin{itemize}
\item intrinsiek: van binnen afkomstig
\end{itemize}
\subsection{De ontworsteling van de logos uit de mythos}
\subsubsection{Logos en mythos}
\begin{itemize}
\item presocratici: voor Socrates
\item gonie: oorsprongsverhaal
\end{itemize}
PLATO: Filosofie is geboren uit verwondering.
\\
Mythe is hybride geheel van diverse interesses (cognitief en controle)
\subsubsection{Filosofie/wetenschap}
Filosofie en wetenschap waren \'e\'en, zeker in het oude Griekenland en nog veel later.
\\
WEBER: ontmythologisering van de wereld.
\subsection{De natuurfilosofen en de leer van de elementen}
Speculatieve theorie\"en kwamen tot stand in stadstaten aan de kusten van de Middellandse Zee.
\\
Physis of wordingsbeginsel.
\\Overgaan van krachten/machten naar principes.
De filosofen van toen:
\begin{itemize}
\item Thales van Milete: de "eerste filosoof".
\item Anaximander
\item Anaximenes
\item Herakleitos
\item Parmenides
\item Zeno
\item Democritos: atomisme (=alle stoffen zijn opgebouwd uit ontelbare minuscule ondeelbare blokjes: atomen)
\item Empedocles: grondlegger van de leer van de vier elementen.
\end{itemize}
Ze zoeken naar een minimaal aantal principes (liefst \'e\'en) in de natuur.
\\
\\
Griekse filosofie geobsedeerd door 2 problemen in de natuur:
\begin{itemize}
\item Eenheid en veelheid
\item Verandering en onveranderlijkheid
\end{itemize}
Op zoek naar orde in de chaos.
\\
\\
Elementaire substanties of elementen, die verborgen substantie is achter alle stoffelijke lichamen:
\begin{enumerate}
\item Alles is water: Thales
\item Alles is aarde:
\item Alles is vuur: Pythagoras \& Herakleitos
\item Alles is lucht: Anaximenes
\item (ether)
\end{enumerate}
Deels logos en deels mythos, er zijn \underline{3 ambigu\"iteiten}:
\begin{enumerate}
\item De redenering is: fysisch $\leftrightarrow$ metafysisch
\\-  De elementen zijn fysisch maar de redering is metafysisch.
\\- zie p78.
\item De redenering is: empirisch en concreet $\leftrightarrow$ abstract
\\- zie p78.
\item De redenering is: ratio gebaseerd $\leftrightarrow$ verbeelding
\\- Bachelard: de vier elementen zijn beelden
\\- zie p79.
\end{enumerate}
Gelijkenissen met mythisch denken:
\begin{itemize}
\item Elementen buiten proporties opgeblazen (speculatief, metafysisch en fysich)
\item Oermachten
\item Verbeelding
\end{itemize}
Verschillen met het mytisch denken:
\begin{itemize}
\item Ervaring, empirischme
\end{itemize}
\section{Alles is getal}
Verborgen kennis van de natuur der dingen ligt niet in de waarneming, maar in het wiskundige inzicht: inzicht in wat juist niet waarneembaar is.
\\
Voorbij de ervaring, achter de ervaring zoeken.
\subsection{Opkomst van de wiskundige opvatting van wetenschap}
Vanaf Pythagoras (560-490). De school van Pythagoras had nog wel steeds mystiek trekje.
\\
Mathematisch rationalisme of mathematisch intelectualisme. Met pythagorisme van groot belang als oorsprong van een grote intellectuele traditie in de wetenschap.
\subsubsection{De ontdekking van wiskundige patronen in de natuur}
\begin{itemize}
\item Orfisme: religieuze beweging en leer in de oudheid.
\item Esotherisch: op de grens met de spirualiteit.
\end{itemize}
\subsubsection{De zuivere wiskunde}
\begin{itemize}
\item Incentive: motivatie door het eerst verkrijgeven van een beloning.
\item Ontologie: religieuze beweging en leer in de oudheid.
\end{itemize}
\subsection{De eerste grondslagencrisis}
Werd veroorzaakt door volgende twee aspecten:
\subsubsection{Crisis van de incommensurabele (=onderling onmeetbare) grootheden}
\begin{itemize}
\item Hypotenusa: schuine zijde van een driehoek.
\end{itemize}
EUDOXUS: tijdgenoot Plato, proportie leer $\rightarrow$ was een hulpmiddel om de rekenkunde toch te kunnen gebruiken in de studie van de ruimte.

\subsubsection{De paradoxen van Zeno}
ZENO van Elea, 5e eeuw v.C.,
\section{Alles is idee}

\subsection{De culturele context}
SOKRATES: kennis is in feite ijdelheid en illusie, eeuwige vragensteller.
\subsection{Platoonse idee\"en}
\textit{"De zuivere idee of vorm is de echte structuur van de werkelijkheid, echter dan het waarneembare en ware grondslag van het waarneembare."}
\\
PYTHAGORAS en PLATO: denken dat we middenin de ervaringsmachine zitten.
\\
Waaktoestand tussen droom en realiteit $\rightarrow$ leer van \underline{vier} kennisgraden:
\begin{enumerate}
\item \underline{De kennis van horen zeggen (overlevering, gerucht):} bvb. mytische verhalen.
\item \underline{De informatie van de zintuigen:} min of meer betrouwbare opnie. Betrouwbaar? Mythe van de grot.
Zintuigen leveren niet louter schijn. 
\\ Plato: geen boeddhist $\rightarrow$ zintuigelijke wereld moet participeren aan het echte.
\\
\\
$\longrightarrow$ \textbf{deze twee vormen samen de \underline{doxa}, de opinie}
\item \underline{De kennis door hypothesen:} gefundamenteerde opinie (dianoia) bvb. de mathematiseerbare kennis van de natuur zoals onderandere: acustica, astronomie, statica, hydrostatica, optica...
\\
Echte kennis hoewel hypothetisch.

\item \underline{De kennis vanuit de ware idee (episteme, gezekerde en gefundeerde kennis):}
gerealiseerd door de filosofie, bvb. de reflectie van de axioma's in de wiskunde. Anders moet je tot in het oneindige blijven bewijzen.
\\
Worden aangetoond door reflectie op betekenis van de begrippen (idee\"en) die erin verbonden worden.
\\
\\
$\longrightarrow$ \textbf{deze twee vormen samen de kennis}
\end{enumerate}
De idee van iets is zijn essentie (wezen, eidos) $\rightarrow$ idee\"enleer; de argumenten hiervoor van Plato:
\begin{enumerate}
\item \underline{Het argument van gelijkenissen:} wat we zien belichaamd nooit de hele wezenheid, we spreken telkens over gelijkenissen.
\\
(Een kat een kat noemen)
\item \underline{Het argument van extensie en intensie (denotatie en connotatie):} katten zijn materialisaties van het kattenwezen dat onverandelijk is. De denotatie is niet genoeg om over vaste betekenissen te beschikken.
\item \underline{Type en token:} er moet onderscheid zijn tussen het algemene begrip en het concrete ding. Tokens zijn niet betekenisvol, betekenis vereist abstracties.
\item \underline{Objectiviteit:} wat objectief is leidt een bestaan dat onafhankelijk is van mijn gedachten erover. De kennis van de essentie is ontdekking.
\\
Plato: kennis is de ontdekking van wat er al was.
\end{enumerate}
\subsection{Mathesis en ethiek}
\begin{itemize}
\item \underline{Het $(4+1)$de argument:}
wiskunde heeft de centrale plaats in de kennis.
Door middel van de wiskunde dat we de idee van die gewone dingen, en van minder gewone dingen, kunnen leren kennen.
Wiskundige ideeën leveren in verhouding tot die dingen de modellen: de kennis van de structuur is de kennis van het model van het ding, het modelding.  Model (=idealisering met alleen de essentiële eigenschappen)
\end{itemize}
Ethiek: modellen voor het menselijk handelen.
\subsubsection*{Dualisme van Plato}
4 grote onderverdelingen van de westerse filosofie:
\begin{enumerate}
\item Epistemologie: kennisleer, probeert antwoord te geven op vragen als: wat kunnen we kennen en hoe? Zijn er soorten kennis? \dots
\item Metafysica: de ultieme aard van de werkelijkheid, wat bestaat allemaal echt?
\item Antropologie: wil metafysica doortrekken voor de mens en zeggen wat mens in wezen is.
\item Ethiek: wil antwoord geven op de vraag wat het goede leven van die $\uparrow$ mens is.
\end{enumerate}
Plato stelt twee dingen radicaal als tegengestelden tegenover elkaar bvb. lichaam en ziel.
\\
\\
\begin{tabular}{p{4cm} c c p{5cm}}
	\hline
	\hline
	epistemologie & metafysica & antroplogie & ethiek \\
	\hline
	\hline
	zintuigelijke waarneming (doxa) & materie & lichaam, zintuigen & onredelijkheid \\
	\hline
	intellectuele kennis (dianoia en episteme) & Idee, geest & ziel & handelen als participatie aan Idee van het goede \\
	\hline
	\hline
\end{tabular}
\subsection{De mathematische fysica}
PLATO: 
Fysica tussen derde en vierde kengraad. 
De Demiurg.
\\
Vijf elementen, het vijde is het etherische element van de bovenmaanse hemelsfeer.
\\
Alles wordt gespiritualiseerd.
\subsection{Kanttekeningen}
\begin{enumerate}
\item Mytische motieven, de leer van Plato is een heilsleer die ons wil oproepen ons af te wenden van de illusie en ons te richten naar het echte zijn.
\item Heilsleer van de ratio heeft dubbel karakter: ethische is hoger dan het wetenschappelijk kenbare $\leftrightarrow$ goede ligt in het verlengde van de kennis.
\item Wat ik ken en inzie dat vind ik zelf niet uit.
\item Plato heeft in type en token geen oog voor situaties waar de gehele betekenis staat of valt niet met het algemene, maar juist met het singuliere concrete.
\end{enumerate}
\section{Alles is leven}
\subsection{Van idee\"en naar essenti\"ele eigenschappen}
Model opvatten als abstract begrip in onze geest? Een soort van mentale connotatie (LANGER)?
\\
\\
PLATO: de objectiviteit van het model vereist dat we het begrip ook opvatten als een Idee met een zelfstandig bestaan buiten ons.
\\
\\
ARISTOTELES: was over het algemeen akkoord met Plato maar NIET met de conclusie van hierboven.
\\
\\
"Als kennis van het object kennis is van de essentie van het object, wil dat daarom zeggen dat die essentie zelf een ideaal objecht op zichzelf is?"
\\
\\
Vandaar de argumenten tegen Plato:
\begin{enumerate}
\item De derde man
\item Materie en vorm: de essentie is de vorm voor zover die in de samenstelling met de materie voorkomt.
\item Eigenschappen zijn geen substanties: eigenschappen bestaan in iets anders. Universalia in re.
\item Substantie heeft een essentie. De essentie is niets meer dan een verzameling essenti\"ele eigenschappen die het ding (naast toevallige eigenschappen) bezit.
\end{enumerate}


\subsection{Biologie, fysica en (een beetje) metafysica}
ARI: Wiskunde heeft niet de centrale plaats in de kennis. 
\\
\\
Graden van abstractie:
\begin{itemize}
\item De fysiche abstractie: mengelvorm van empirisch en ratioeel denken die past bij natuurobjecten (levende dingen).
\item De wiskundige abstractie: past bij wiskundige objecten en natuurobjecten die een wiskundige regelmaat vertonen zoals het "bovenmaanse"; bvb. de astronomie.
\item De metafyische abstractie: datgene dat alle objecten aan kenmerken moeten bezitten om \"uberhaupt te bestaan.

\end{itemize}

De natuurwetenschapen splitsen zich in twee:
\begin{enumerate}
\item Wiskundige disciplines: bovenmaanse natuur.
\item Levenswetenschappen en ervaringswetenschappen: ondermaanse natuur.
\end{enumerate}
$\rightarrow$ de band tussen wiskundige en ervargingswetenschappen gaat voor vele eeuwen verloren.
\subsection{Logica}
Syllogisme: de eerste echte logische theorie door Ari.
\subsection{Moraal en praktische filosofie}
Doel en moraal van het leven zijn dezelfde: vorm van welslagen vinden dat de gelukkige mens kenmerkt. Gelukkige leven (eudaimonia) is een kwestie van deugd.
\\
\\
Phronesis de meest typische aristotelische deugd: redelijke inzicht in de aard der dingen dat in staat is ook evenwicht te bereiken in het handelen in overeenstemming met die aard der dingen.
\subsection{Kanttekeningen}
'Trouble with Ari': door zijn empirisch realisme bereik van wiskunde fel ingeperkt.
\section{Alles is bedoeld}
\subsection{De kentering}
Na Plato en Ari: theoretische filosofie over zijn hoogtepunt heen.
\\
Romeinen vonden niets nieuw uit in de filosofie en gebruikten enkel de wiskunde van de Grieken. 
\\
Wel vonden ze de technische interesse opnieuw uit.
\subsection{Bilan van het antieke denken. De derde factor}
\subsubsection{Twee concepties}
\begin{enumerate}
\item Filosofie als uiting van zuiverste kennisdrang.
\item Filosofie gericht op 'levenswijsheid".
\end{enumerate}
\subsubsection{De derde factor}
\begin{enumerate}
\item[3.]Zin-interesse.
\end{enumerate}
Ontstaan alliantie tussen de kennisinteresse en de ‘derde factor’.
\subsubsection{Twee stromingen}
De eerste conceptie in paragraaf \"e\"en moet nog onderverdeeld worden in:
\begin{enumerate}
\item Epirische onderzoekstraditie.
\item Rationalistische onderzoekstraditie.
\end{enumerate}
\subsection{De christelijke middeleeuwen}
Niet tegen de rede, maar op zoek naar verzoening ermee.
Filosofische basis: neoplatoonse school: PLOTINUS, PROCLUS (300 n.C.) Eerste synthese AUGUSTINUS.
ARI in de arabische cultuur. Dus naast wiskundige wetenschappen (begrijpbaar voor Christenen) ook: biomed. ervaringswetenschappen, anatomie, biologie, fysica en geneeskunde.
Westen onder de indruk, vertaalcampagne: ARI, grote Arabische denkers (AVICENNA en AVERROES).
1250: eerste universiteiten strijd: platonisten vs. Averroïsten.
-> Synthese door Thomas van AQUINO
\\
\\
God heeft de natuurlijke wereld zo willen scheppen dat hij voor mensen met hun natuurlijke rede kenbaar is.
\\
\\
APOTHEOSE van de ALLIANTIE tussen KENNIS en ZINGEVING.
\subsection{Doeloorzakelijkheid}
Goddelijk scheppingsplan biedt natuurlijk inspiratiekader voor de doeloorzakelijkheid waar ARI op hamerde. (De schepping is duidelijk bedoeld.)
\\
4 oorzaken volgen Ari:
\begin{enumerate}
\item Materi\"ele oorzaak: materiaal om iets te bouwen.
\item Effici\"ente oorzaak: handeling van de maker.
\item Formele oorzaak: de vormen of essenties die de materie kan aannemen.
\item Doeloorzaak: de bestemming van het eindproduct.
\end{enumerate}
Finalistisch of teleologische denkwijze: er steeds voor alles een doeloorzaak, een telos of doel waarnaar het natuurzijnde streeft.
