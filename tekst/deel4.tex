\section*{\centering \underline{Deel 4: Hoe overleven we de wetenschap?}}
\section{De tafel van Eddington}
\subsection{Het wetenschappelijke en manifeste beeld}
Arthur EDDINGTON (1882-1944): \textit{Stel dat de atoomtheorei klopt, dan zijn tafels en stoelen in feite niets dan een hoop atomen en nog kleinere partikels dwarrelend in een lege ruimte?!}
\\
$\rightarrow$ De ervaring is dan een subjectieve machine, een fabriek van illusies bovenop de echte onderliggende realiteit met haar andere, onzichtbare gezicht.
\\
\\
EDDINGTON zocht oplossing voor dit dilemma:
\\
Aannemen van complementarieit of correspondentie tussen beide beschrijvingsniveaus.
\\
\\
Wilfrid SELLARS (1912-1989): veralgemeent probleem van EDDINGTON $\rightarrow$ het gaat niet alleen over macroscopische objecten, maar de gehele leefwereld.
\\
Twee beelden van de realiteit:
\begin{enumerate}
\item Wetenschappelijke beeld
\item Manifeste beeld
\end{enumerate}
Volgens SELLARS moet je kiezen tussen de twee beelden:
\begin{enumerate}
\item De argumenten pleiten voor een vorm van realisme, ze willen weten het nu echt zit, los van voorstellingen: dan kan met niet evenveel geloof hebben in beid.
\item Beide beelden maken een totaalclaim.
\item Het succes-argument: wetenschap is een \textit{succes story}, het manifeste beeld is dat niet. In alle gebiede buite dat van beweste menselijke ervaring heeft wetenschappelijke beeld gewonne van het manifeste. SELLARS zegt dat de wetenschappen van de mens onderdeel moeten worden van de natuurwetenschappen.
\end{enumerate}
De argumenten roepen veel vragen op, vooral rond de premissen realisme en totaalbeelden:
\begin{enumerate}
\item[i.] Je verwerpt beide premissen:
\newtheorem*{def1}{Pragmatisch pluralisme}
\begin{def1}
Wie bv. meent dat wetenschappelijke theorie\"en de beste instrumenten zijn om aan fysica, chemie en biologie te doen, maar het manifeste common sense beeld (inclusief folk pychology) het beste instrument om aan psychologie, esthetica, ehtica, pedagogiek en didactiek, talen, geschiedenis, dagelijkse communicatie enz. te doen.
\end{def1}

\item[ii.] Je aanvaardt totaalclaims maar verwerpt realisme:
\newtheorem*{def2}{Instrumentalisme}
\begin{def2}
Wie bv. meent dat wetenschap in het algemeen overal het beste instrument is om kennis te hebben (totaalclaim), maar daarom niet als een letterlijk beeld van de realiteit gezien moet worden. Of andersom: dat het manifeste overal het beste instrument levert om de wereld te bespreken, maar daarom nog niet letterlijk waar.
\end{def2}

\item[iii.] Je aanvaardt realisme maar verwerpt totaalclaims:
\newtheorem*{def3}{Realistisch pluralisme}
\begin{def3}
Wie bv. meent dat wetenschap het beste realisitsche beeld is om fysica, biologie, enz. te doen \'en het manifeste common sense beeld (inclusief folk psychology) het beste realisitsche beeld om aan psychologie, esthetica, ethica, didactiek, ... te doen.
\end{def3}

\item[iv.] Beide premissen aanvaarden maar die van bv. realisme beperkter:
\newtheorem*{def4}{Compatibilisme I}
\begin{def4}
Wie bv. meent dat wetenschap het beste realistische beeld geeft van alles, en common sense het beste instrument om minstens een deel van de fenomenen te bespreken (bv. in de psychologie, enz.).
\end{def4}
\item[v.] Beide premissen aanvaarden maar die van bv. realisme beperkter:
\newtheorem*{def5}{Compatibilisme II}
\begin{def5}
Wie bv. zegt: wetenschappelijke psychologie is wellicht het beste beeld, realistisch of misschien zelfs ook los van realisme, als het op de objectieve (zo objectief mogelijke) kennis aankomt van de geest; maar common sense is het beste, zelfs het enige beeld dat ons toelaat andere da zuiver cognitieve taal te spreken over de geest enz. Zo bv. om over de zin (of onzin!) van dingen te spreken, om over mensen, menselijke communicatie en verhoudingen te spreken, om over cultuur te spreken: we kunnen bv. niet over mensen als mensen (dus als \'e\'er dan machines) spreken zonder ze als personen te beschouwen, en we kunnen dan weer niet over personen spreken zonder ze beliefs \& desires toe te schrijven.
\end{def5}
\end{enumerate}

\textbf{Overzicht: mogelijke alternatieven voor SELLARS}
\\
\begin{tabular}{l c c}
\hline
realisme totaalbeeld & - & +
\\
- & pragmatisch pluralisme (i) & instrumentalisme (ii)
\\
+ & realistisch pluralisme (iii) & compat. I (iv) compat. II (v)
\\
\hline
\end{tabular}
\subsection{Metafysisch realisme en science freaks}
De idee dat je over de wereld ook voor zover die zich gedeeltelijk aan de waarneming onttrekt een beeld, een theorie moet opstellen, die je op zijn beurt voor waar of onwaar moet houden.
\\
Volgens deze vorm van wetenschappelijk realisme wordt wetenschap de 'echte metafysica':
\begin{enumerate}
\item Realisten: er bestaat objectieve kennis over alles.
\\
Antirealisten: er bestaan enkel objectieve kennis voer waarneembare realiteit.
\item Het samenspel van ratio en experimentele empirie geeft betekenis aan elke objectiviteit.
\item Samenspel van kennis en controle. Wetenschap de absolute conception of reality, opgesteld van een volledig extern standpunt. Als dit zo is, dan is verandering en beheersing geen onderdeel van haar project. Elke menselijke interesse is er vreemd aan. Wetenschapper wordt goddelijk, ziet alles vanuit pure theoria
\\
\\
$\Rightarrow$ Hoe overleven we zo'n wetenschap?
\end{enumerate}
\section{De bessen van mijn jeugd}
\subsection{Twee echtheden, drie graden van belichaming}
LANGER: opvatting volgens dewelke we alleen door instrumentele interesses worden gedreven, moeilijk in overeenstemming is te brengen met onze gerichtheid op symbolen.
Iets méér dan een interesse om te weten. (Symbolische erkenning bij voorbeeld met de biologische vader)
Het ‘echte kennen’ is weten hoe het is, los van het verlangen dat het zus of zo zou zijn.
\\
\\
Fundamentele interesses, symoblisch op twee gronden:
\begin{enumerate}
\item Menselijk symbolisch systeem van taal.
\item Gebruik van symbolen in een tegelijk striktere en bredere, maar alleszins pertinente zin. Symbolen hebben een intrinsiek belang.
\end{enumerate} 

Toegepast op de fundamentele interesses:
\begin{enumerate}
\item Kennis wetenschap) als techniek (controle, beheersing) en zingeving zijn symbolisch: zij hebben de taal nodig om gecommuniceerd en van generatie op generatie doorgegeven te worden.
\item Verschil tussen de drie interesses:
\begin{itemize}
\item Zwak belichaamde betekenissen: techniek maakt weinig of geen gebruik van strikte symbolen. 
	Wetenschap: méér sprake van intrinsieke symbolen.
\item Matig belichaamde betekenissen: kennisinteresse is in de eerste plaats intrinsiek (weten uit een drang om te weten) LAPLACE, PRIGOGINE
De betekenis van wetenschappelijke stellingen kun je in verschillende graden van exactheid en moeilijkheid omschrijven of benaderen. 
\item Sterk belichaamde symbolen: derde interesse (zingeving), niet mogelijk afstand te doen van de symbolen. 
\end{itemize}
\end{enumerate}
De betekenis van sterk belichaamde symbolen is evocatief en ondoorzichtig, die van informatieve taal expliciet en transparant. 
\subsection{De drie interesses: contaminatie en behoud van verschil}
Moderne wetenschap $\rightarrow$ alliantie met de techniek. De vraag naar de zin naar een aparte sfeer verwezen. 
\\
De rationaliteit en haar grenzen. BURMS en DE DIJN
\\
Wetenschap is een empirische praktijk die in het verlengde ligt van de common sense. Geen re\"ele bedreigingen kunnen uitgaan van de ene interesse voor de andere. Interesses kunnen elkaar wel in het gedrang brengen door een soor van verwarring in de hoofden van de mensen. 
Ook de mythe is een kluwen geweest van interesses. HORTON
Co-existentie.
\subsection{... en het magisch symbolisme?}
Waarom geïnteresseerd zijn in iets dat evengoed gesimuleerd kan worden?
\\
Geen empirisch verschil, dan kan de fascinatie (en ontgoocheling) enkel een kwestie zijn van ‘magie’. Verschil zit in de eigen symbolische houding. 
\\
STENDHAL, ALAIN-FOURNIERS en LEVI
